\documentclass[a4paper, 14pt]{extarticle}
\usepackage[english, russian]{babel}
\usepackage[utf8x]{inputenc}
\usepackage{fullpage}
\usepackage{indentfirst} % Первый абзац в разделе тоже с красной строки
\usepackage{cmap} % для кодировки шрифтов в pdf (чтобы не было крокозябры при копировании из pdf )
\usepackage{graphicx} % для вставки картинок
\graphicspath{{res/}} % Путь к папке с картинками
\sloppy % Включение переноса слов в тексте

\title{Фреймворк для интерактивного моделирования распределенных задач с запаздыванием на вычислительном кластере}
\author{Фролов Даниил Александрович}

% Для красивой вставки исходного программного кода 
\usepackage{listings}
\usepackage{color}

\definecolor{mygreen}{rgb}{0,0.6,0}
\definecolor{mygray}{rgb}{0.5,0.5,0.5}
\definecolor{mymauve}{rgb}{0.58,0,0.82}

% Параметры раскрски исходного кода программы 
\lstset{ %
	language = Java,
	extendedchars=\true, %Чтобы русские буквы в комментариях были
	%inputencoding=cp1251,
	%commentstyle=\itshape,
	%stringstyle=\bf,
  backgroundcolor=\color{white},   % choose the background color
  basicstyle=\footnotesize,        % size of fonts used for the code
  %basicstyle=\ttfamily\fontsize{11pt}{11pt}\selectfont,
  breaklines=true,                 % automatic line breaking only at whitespace
  captionpos=b,                    % sets the caption-position to bottom
  commentstyle=\color{mygreen},    % comment style
  escapeinside={\%*}{*)},          % if you want to add LaTeX within your code
  keywordstyle=\color{blue},       % keyword style
  stringstyle=\color{mymauve},     % string literal style
}

\usepackage{hyperref} % Для добавления ссылок в тесте

% Настройка цветов для ссылок
\hypersetup{
colorlinks = true,
linkcolor = black,
pagecolor = black,
urlcolor = blue, 
citecolor = black
}

% Математика
\usepackage{amssymb} % For use "mathbb" function
\usepackage{amsmath}
\usepackage{amsthm}
\usepackage{mathrsfs}
\newcommand{\La}{\mathscr{L}} % Функция Лагранжа
\newcommand{\ls}{{ℓ}} % Красивая l, чтобы легче было отличить от i, 1 b других палок
\providecommand{\norm}[1]{\lVert#1\rVert} % Норма вектора : ||w||
\newcommand{\dpt}[1]{\left\langle#1\right\rangle} % dot product using brackets
\newcommand{\brackets}[1]{\left(#1\right)} % Обернуть скобками автоматического размера
%\newcommand{\dpts}[2]{#2 \langle#1 #2 \rangle} % dot product using brackets with manual size
\newcommand{\R}{\mathbb{R}} % beautiful R for R^n labels
\newcommand{\il}{i = 1, \ldots, \ls} % writes i = 1, ..., l
\newcommand{\ili}{\quad i = 1, \ldots, \ls} % writes i = 1, ..., l with indent in begin
\newcommand{\sumil}{\sum_{i=1}^{\ls}} % Сумма по i, которая изменяется от 1 до l
\newcommand{\minl}{\min\limits} % min with limits under "min" label
\newcommand{\maxl}{\max\limits} % max with limits under "max" label

% Окружение для теорем, определений и т.д.
\theoremstyle{definition}
\newtheorem{definition}{Определение}
\newtheorem{theorem}{Теорема}
\newtheorem{example}{Пример}

% Поправить стиль отрисовки формул (особенно актуально для сумм https://ru.sharelatex.com/learn/Display_style_in_math_mode)
\everymath{\displaystyle}

%\bibliographystyle{unsrt} % упорядочить список использованной литературы по порядку упоминания их в тексте
\bibliographystyle{utf8gost705u}
%\bibliographystyle{utf8gost71u}

\usepackage[labelfont=bf, labelsep=space]{caption} % Делаем надписи "Рис.1" под рисунками жирными и без двоеточия.
\usepackage[top=20mm, bottom=20mm, left=30mm, right=20mm
, nohead % Убрать расстояние для верхних колонтикулов
%, nofoot % Убрать расстояние для нижних колонтикулов
]
{geometry} % Размер полей у старницы
\setlength{\parindent}{1.25cm} % Размер интервала для абзацев 
\usepackage{setspace}
\singlespacing % одинарный интервал
\renewcommand\normalsize{\fontsize{14}{16.8pt}\selectfont} % Сделать размер шрифта 14

\usepackage{caption} % подписи к рисункам в русской типографской традиции
\DeclareCaptionFormat{GOSTtable}{#2#1\\#3\vspace*{-\baselineskip}}
\DeclareCaptionLabelSeparator{fill}{\hfill}
\DeclareCaptionLabelFormat{fullparents}{\bothIfFirst{#1}{~}#2}
\captionsetup[table]{
     format=GOSTtable,
     %font={footnotesize},
     labelformat=fullparents,
     labelsep=fill,
     labelfont=normal,
     textfont=bf,
     justification=centering,
     singlelinecheck=false
     }

% Начинать каждую главу с новой страницы
\usepackage{titlesec}
\newcommand{\sectionbreak}{\clearpage}

\newcommand{\newsection}[1]{
	\cleardoublepage
	\phantomsection
	\section*{#1}
	\addcontentsline{toc}{section}{#1}
}

\newcommand{\newsubsection}[1]{
	%\newpage
	\phantomsection
	\subsection*{#1}
	\addcontentsline{toc}{subsection}{#1}
}

% Позволяет подсчитывать число страниц в документе с помощью комманды \pageref*{LastPage}
\usepackage{lastpage}
%\pretolerance 10000

\begin{document}
\setcounter{tocdepth}{3}

% Позволяет подсчитывать число рисунков, таблиц и глав(section) в документе
\newcounter{totfigures}
\newcounter{tottables}
\newcounter{totsections}
\makeatletter
\AtEndDocument{%
	\addtocounter{totfigures}{\value{figure}}%
	\addtocounter{tottables}{\value{table}}%
	\addtocounter{totsections}{\value{section}}%
	\immediate\write\@mainaux{%
		\string\gdef\string\totfig{\number\value{totfigures}}%
		\string\gdef\string\tottab{\number\value{tottables}}%    
		\string\gdef\string\totsections{\number\value{totsections}}%  
	}%
}
\makeatother


%Титульный лист
{
\thispagestyle{empty}

\begin{center}
	
	Министерство образования и науки Российской Федерации\\[0.3cm]
	Государственное образовательное учреждение\\
	высшего профессионального образования\\
	"<Ярославский государственный университет им. П.Г. Демидова">\\
	(ЯрГУ)\\[0.3cm]
	
	Кафедра дискретного анализа
	
	\bigskip
	
	\hspace{15em}"<Допустить к защите">
	
	\begin{flushright}
		Заведующий кафедрой\par
		д. ф.-м. н., профессор\par
		\underline{\hspace{3.2cm}}Бондаренко\,В.\,А.\par
		"<\underline{\hspace{0.5cm}}">\underline{\hspace{3.4cm}}2017 г.\par
	\end{flushright}
	
	\bigskip
	
	{\textbf
		{\textit
			{Магистерская диссертация}
		}
	}
	\\
	по направлению 01.04.02 Прикладная математика и информатика
	
	\bigskip
	
	{\bf
		Фреймворк для интерактивного моделирования распределенных задач с запаздыванием на вычислительном кластере 
	}
\end{center}

\medskip

\begin{flushright}
	Научный руководитель\par
	д. ф.-м. н., профессор\par
	\underline{\hspace{3.5cm}}Глызин\,С.\,Д.\par
	"<\underline{\hspace{0.8cm}}">\underline{\hspace{3.5cm}}2017 г.\par
\end{flushright}

\bigskip 

\begin{flushright}
	Студент группы ИВТ-21-МО\par
	\underline{\hspace{2.5cm}}Фролов\,Д.\,А.\par
	"<\underline{\hspace{0.8cm}}">\underline{\hspace{3.5cm}}2017 г.\par
\end{flushright}

\vspace{\fill}

\begin{center}
	Ярославль 2017
\end{center}

\clearpage
}



%Реферат
\section*{Реферат}
123
%\noindent Объем \pageref*{LastPage}~с., 
%\totsections~гл., \totfig~рис., 
%7 источников, 2 прил. 
%
%\noindent\textbf{OpenMP, MPI, CUDA, параллельные вычисления, диффузионные задачи}
%
%
%\par Работа описывает реализацию части программного комплекса для моделирования диффузионных задач, отвечающую за параллельные вычисления. Рассматриваются теоретические основы численного решения задач "<реакция-диффузия">, а также формулируются требования к программного комплексу.
%
%\par В работе приводится описание архитектуры приложения и используемых классов, а также рассматривается алгоритм параллельных вычислений, применяемый в программном комплексе.
%
%\par Автором работы проведен анализ производительности на различных комбинациях устройств.




%Содержание
\tableofcontents





%Введение
\newsection{Введение}

Развитие современной науки постоянно ставит перед исследователями задачи, для решения которых зачастую необходимо создавать различные математические модели, а также проводить их анализ. К таким моделям, например, можно относить системы <<реакция-диффузия>>. Фактически мы говорим о распределенных в пространстве задачах, часть из которых дополнительно могут быть усложнены наличием запаздывания по времени в правой части системы дифференциальных уравнений\cite{GlyzinSD_1}.

Системы вида <<реакция--диффузия>> входят в класс нелинейных динамических систем. В подобных задачах неоднородные колебательные режимы обуславливаются диффузионной частью. Подобные модели часто можно встретить при решении физических задач, биологических, а кроме того при рассмотрении задач популяционной биологии.

Опираясь на вышеизложенные аргументы, можно сказать, что вопрос автоматического моделирования задач подобного рода является актуальной проблемой. Нельзя упускать из внимания то факт, что продукты, позволяющие моделировать конкретные физические, биологические или химические процессы, то есть решать конкретную краевую задачу, уже появлялись ранее.

Вычислительная техника и информатика в наше время дает возможность решать проблему сложности моделирования подобных задач на совершенно ином уровне абстракции, на более высоком. Фактически это значит, что созданных фреймворк позволяет моделировать практически любую задачу, заданную пользователем. Конечно, если пользователь знает конкретную математическую модель, которая будет описывать интересующий его процесс.

Как уже говорилось ранее, подобный программный комплекс позволить решать уже не одну единственную, заранее заданную задачу, в которой доступно лишь изменение ограниченного числа параметров, а целый спектр, с первого взгляда непохожих друг на друга задач. Все они будет объединены только лишь общим видом допустимой системы.

Подобному программному комплексу необходимо обладать рядом качеств. В первую очередь, это широкий спектр возможностей для <<настройки>> решаемой задачи, а именно, возможность для конечного пользователя задавать множество параметров, таких как: математическая модель решаемой задачи, описание области, в которой происходит моделируемый процесс, точность вычислений, как с точки зрения пространства так и со стороны шага по времени, кроме того, необходима возможность выбора метода численного моделирования из предложенных и другие настройки. Во~-вторых, фреймворк должен показывать эффективную работу на современных вычислительных кластерах, при использовании нескольких физически разных машин.

С учетом вышеизложенного перед автором работы была поставлена задача разработки фреймворка, моделирования диффузионных задач на вычислительных кластерах. Решение этой задачи лежит в области распределенных вычислений, то есть применения современных методов распараллеливания для повышения производительности вычислений.




%Первая глава
\section{Постановка задачи}

\subsection{Теоретические основы решения задач <<реакция--диффузия>>}
Методы, применяемые при разработке фреймворка рассмотри на примере задач <<реакция--диффузия>>.

В общем случае рассматриваемые системы дифференциальных уравнений представимы в виде следующей краевой задачи:
$$\frac{\partial u}{\partial t} = D \frac{\partial^2 u}{\partial x^2} + F(u);$$
$$\left.{\frac{\partial u}{\partial x}} \right|_{x=0} = \left.{\frac{\partial u}{\partial x}} \right|_{x=1} = 0, F(0) = 0.$$

Для выполнения численного моделирования любым методом необходимо использовать подход, который подразумевает приближение оператора Лапласа из правой части системы дифференциальных уравнений его разностными аналогами. В представленном случае преобразование будет выглядеть следующим образом.
$$x_i = \bigtriangleup(i + \frac{1}{2}), \bigtriangleup = \frac{1}{N},$$
где N - количество отсчетов на оси.
$$\left.{\frac{\partial^2 u}{\partial x^2}}\right|_{x=x_i} = \frac{u_{i-1} - 2u_i + u_{i+1}}{\bigtriangleup^2};$$
$$\dot u_i = D\, \frac{u_{i-1} - 2u_i + u_{i+1}}{\bigtriangleup^2} + F(u_i);$$
$$u_0 = u_1, u_{N+1} = u_N, i = \overline{1, N}.$$

В примере фигурировала одномерная задача с использованием производной второго порядка в правой части, но аналогичный подход можно реализовать для задач большей размерности и с производными большего порядка. Наличие запаздывания в условиях задачи не изменяет алгоритм, так как подобное усложнение никак не влияет на преобразования, выполняемое для оператора Лапласа.

После выполнения требуемых операций получается система обыкновенных дифференциальных уравнений.  Стандартные численные сеточные методы, такие как метод Эйлера, метод Рунге--Кутты, Дормана--Принса, позволяют успешно моделировать задачи\cite{Thomas}. 

Ниже представлены несколько классических численных методов, которые реализованы в рассматриваемом программном продукте. Каждый из них позволяет решать начальную задачу Коши для систем обыкновенных дифференциальных уравнений.

Приведем схемы работы нескольких стандартных методов численного решения начальной задачи Коши для систем обыкновенных дифференциальных уравнений. Метод Эйлера считается наиболее простым из подобных методов. Он относится к явным, одношаговым методам первого порядка точности и фактически основан на аппроксимации интегральной кривой кусочно-линейной функцией. Упрощенно метод Эйлера может быть представлен в следующем виде.

\par Пусть дана начальная задача Коши
$$\dot y = f(t, y)$$
и некоторое начальное условие
$$y(t_0) = y_0.$$

\par Тогда новое значение можно получить, с помощью предыдущего, используя следующее равенство:
$$y_{n+1} = y_n + \bigtriangleup t \cdot f(t, y_n).$$

\par Второй метод, который может быть использован для решения поставленной задачи, -- метод Рунге-Кутты. Данный метод кратко описывается следующей схемой.

\par Пусть дана такая же начальная задача Коши, что и для метода Эйлера. Тогда каждую последующую точку решения можно вычислить с помощью предыдущей, используя следующее равенство:
$$y_{n+1} = y_n + \sum_{i=1}^s b_i k_i.$$

\par Коэффициенты $k_i$ вычисляются, исходя из следующих соотношений:
$$k_1 = hf(t_n, y_n),$$
$$k_2 = hf(t_n + c_2 h, y_n + a_{21}k_1),$$
$$\cdots$$
$$k_s = hf(t_n + c_s, y_n + a_{s1}k_1 + a_{s2}k_2 + \cdots + a_{s,s-1}k_{s-1}).$$

\par Количество коэффицентов $k_i$ определяет порядок метода: к примеру, для четырехстадийной схемы Рунге-Кутты необходимо вычислить четрые таких коэффициента.

\par Для параметров $a_{ij}$, $b_i$, $c_i$, которые используются при расчеты $k_i$, должны выполняться следующие равенства:
$$\sum_{j=1}^{i-1}{a_{ij}} = c_i,$$
$$\sum_{j=1}^s{b_i} = 1,$$
$$0 \le (b_i, c_i, a_{ij}) \le 1,$$
$$b_i > 1.$$

\par Именно параметры $a_{ij}$, $b_i$ и $c_i$ задают конкретный метод Рунге-Кутты.

\par Как правило, для решения дифференциальных уравнений используются схемы невысоких порядков точности, поэтому в нашем случае достаточно воспользоваться четырехстадийной реализацией метода Рунге-Кутты, то есть взять $s = 4$.

\par Для решения задачи "<реакция-диффузия">, как уже было сказано выше, также может быть использован метод Дормана-Принса, который относится к классу методов Рунге-Кутты. По сравнению с последним он является более точным за счет большего числа стадий вычисления нового состояния. Среди особенностей метода также стоит отметить изменяемость шага вычислений.

\par Очевидно, что применение того или иного метода, оказывает влияние на точность и скорость производимых вычислений и, естественно, на конечный результат. В связи с этим, выбор конкретного метода решения следует оставить за пользователем.



\subsection{Основные требования к приложению}
123



%Заключение
\newsection{Заключение}
123






%Список литературы
%\begin{thebibliography}{99}
%\addcontentsline{toc}{section}{Список литературы}
%
%\bibitem{GlyzinSD1}
%\textit{Глызин С.\,Д., Шокин П.\,Л. } 
%{Диффузионный хаос в задаче «реакция–диффузия» c гантелеобразной областью определения пространственной переменной. 
%М.: Модел. и анализ информ. систем, 2013. -- 43--57 с. }
%
%\bibitem{GlyzinSD2}
%\textit{Глызин С.\,Д. } 
%{Размерностные характеристики диффузионного хаоса. 
%М.: Модел. и анализ информ. систем, 2013. -- 30--51 с. }
%
%\bibitem{AntonovOpenMP}
%\textit{Антонов А.\,С. } 
%{Параллельное программирование с использованием OpenMP. 
%М.: Изд-во МГУ, 2009. -- 77 с. } 
%
%\bibitem{SandersCUDA}
%\textit{Сандерс Дж., Кэндрот Э. } 
%{Технология CUDA в примерах. Введение в программирование графических процессоров. 
%М.: ДМК Пресс, 2013. -- 232 с. } 
%
%\end{thebibliography}

\cleardoublepage
\phantomsection
\addcontentsline{toc}{section}{Список литературы}
\bibliography{bibliography/refs}


\newpage

\begin{flushright}
Приложение А
\end{flushright}
\addcontentsline{toc}{section}{Приложение А}

\begin{center}
\textbf{Класс Block}
\end{center}

\begin{lstlisting}
class Block {

protected:
	Solver* mSolver;

	int dimension;

	int xCount;
	int yCount;
	int zCount;

	int xOffset;
	int yOffset;
	int zOffset;

	unsigned short int* mCompFuncNumber;

	int cellSize;
	int haloSize;

	int deviceNumber;

	int nodeNumber;

	int* sendBorderInfo;
	std::vector<int> tempSendBorderInfo;

	int* receiveBorderInfo;
	std::vector<int> tempReceiveBorderInfo;

	double** blockBorder;
	int* blockBorderMemoryAllocType;
	std::vector<double*> tempBlockBorder;
	std::vector<int> tempBlockBorderMemoryAllocType;

	double** externalBorder;
	int* externalBorderMemoryAllocType;
	std::vector<double*> tempExternalBorder;
	std::vector<int> tempExternalBorderMemoryAllocType;

	int countSendSegmentBorder;
	int countReceiveSegmentBorder;

	void freeMemory(int memory_alloc_type, double* memory);
	void freeMemory(int memory_alloc_type, int* memory);

	virtual void prepareBorder(int borderNumber, int zStart, int zStop, int yStart, int yStop, int xStart, int xStop) = 0;

	virtual void computeStageCenter_1d(int stage, double time) = 0;
	virtual void computeStageCenter_2d(int stage, double time) = 0;
	virtual void computeStageCenter_3d(int stage, double time) = 0;

	virtual void computeStageBorder_1d(int stage, double time) = 0;
	virtual void computeStageBorder_2d(int stage, double time) = 0;
	virtual void computeStageBorder_3d(int stage, double time) = 0;

	virtual void createSolver(int solverIdx,
		double _aTol, double _rTol) = 0;

	virtual double* getNewBlockBorder(Block* neighbor,
		int borderLength, int& memoryType) = 0;
		
	virtual double* getNewExternalBorder(Block* neighbor,
		int borderLength, double* border, int& memoryType) = 0;

	func_ptr_t* mUserFuncs;
	initfunc_fill_ptr_t* mUserInitFuncs;
	double* mParams;
	int mParamsCount;

public:
	Block(int _dimension, int _xCount, int _yCount, int _zCount,
			int _xOffset, int _yOffset, int _zOffset,
			int _nodeNumber, int _deviceNumber,
			int _haloSize, int _cellSize);

	virtual ~Block();

	virtual bool isRealBlock() = 0;

	virtual void initSolver() { return; }

	void prepareStageData(int stage);

	void computeStageBorder(int stage, double time);
	void computeStageCenter(int stage, double time);

	void prepareArgument(int stage, double timestep );

	double getSolverStepError(double timeStep) {
		return mSolver != NULL ? mSolver->getStepError(timeStep) : 0.0;
	}

	void confirmStep(double timestep);
	void rejectStep(double timestep);

	virtual int getBlockType() = 0;

	virtual void print() = 0;

	virtual void getCurrentState(double* result) = 0;

	int getXCount() { return xCount; }
	int getYCount() { return yCount; }
	int getZCount() { return zCount; }

	int getXOffset() { return xOffset; }
	int getYOffset() { return yOffset; }
	int getZOffset() { return zOffset; }

	int getGridNodeCount();
	int getGridElementCount();

	int getDeviceNumber() { return deviceNumber; }
	int getNodeNumber() { return nodeNumber; }

	int getDimension() { return dimension; }

	void setFunctionNumber(unsigned short int* functionNumberData ) { return; }

	double* addNewBlockBorder(Block* neighbor, int side,
		int mOffset, int nOffset, int mLength, int nLength);
		
	double* addNewExternalBorder(Block* neighbor, int side,
		int mOffset, int nOffset,
		int mLength, int nLength, double* border);

	virtual void moveTempBorderVectorToBorderArray() = 0;

	virtual void loadData(double* data) = 0;
};
\end{lstlisting}










\newpage

\begin{flushright}
Приложение Б
\end{flushright}
\addcontentsline{toc}{section}{Приложение Б}

\begin{center}
\textbf{Функция передачи данных между узлами}
\end{center}

\begin{lstlisting}
// Function for sending and receiving data
void Interconnect::sendRecv(int locationNode) {
	// Devices are located on the same node
	if(locationNode == sourceLocationNode && locationNode == destinationLocationNode)
		return;
		
	// This thread should send the information
	if(locationNode == sourceLocationNode) {
		MPI_Isend(sourceBlockBorder, borderLength, MPI_DOUBLE, destinationLocationNode, 999, MPI_COMM_WORLD, request);
		flag = true;
		return;
	}

	// This thread should receive the information
	if(locationNode == destinationLocationNode) {
		MPI_Irecv(destinationExternalBorder, borderLength, MPI_DOUBLE, sourceLocationNode, 999, MPI_COMM_WORLD, request);
		flag = true;
		return;
	}
}
\end{lstlisting}

\end{document}